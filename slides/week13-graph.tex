\documentclass[aspectratio=169, 14pt]{beamer}
\usepackage[utf8]{inputenc}
\usepackage[english]{babel}
\usepackage{tipa}
\usepackage{graphicx}
\usepackage{transparent}
\usepackage[ruled, lined, linesnumbered, commentsnumbered]{algorithm2e}
\usepackage{pgfplots}
\newcommand\mycommfont[1]{\small\ttfamily\textcolor{blue}{#1}}
\SetCommentSty{mycommfont}
\renewcommand{\thealgocf}{}
\usepackage{setspace}
\usepackage{tikz}
\usetikzlibrary{matrix,backgrounds}
\usetikzlibrary{arrows}
\usetikzlibrary {arrows.meta}
\usetikzlibrary{calc,shadows.blur,fit,positioning}
\usetikzlibrary{shapes.multipart,chains}
\usepackage{minted}
\usepackage{fontawesome5}
\usepackage{booktabs}
\usepackage{caption}
\usepackage{bookmark}
\usepackage{hyperref}
\hypersetup{
	colorlinks=true,
	linkcolor=blue,
	filecolor=magenta,
	urlcolor=cyan,
}
\urlstyle{same}
\usetheme{metropolis}
\metroset{block=fill}
\usecolortheme{default}
\definecolor{darkmidnightblue}{rgb}{0.0, 0.2, 0.4}
\definecolor{LightGray}{gray}{0.9}


%------------------------------------------------------------
%This block of code defines the information to appear in the
%Title page
\title[Data Structures] %optional
{Data Structures}

\subtitle{Graph}

\author[CHEN Zhongpu] % (optional)
{CHEN Zhongpu}

\institute[] % (optional)
{
	School of Computing and Artificial Intelligence \\
	\href{mailto:zpchen@swufe.edu.cn}{zpchen@swufe.edu.cn}
}

\date[] % (optional)
{SWUFE, Fall \the\year{}}

%End of title page configuration block
%------------------------------------------------------------


%------------------------------------------------------------
%The next block of commands puts the table of contents at the 
%beginning of each section and highlights the current section:

% \AtBeginSection[]
% {
%   \begin{frame}
%     \frametitle{Table of Contents}
%     \tableofcontents[currentsection]
%   \end{frame}
% }
%------------------------------------------------------------


\begin{document}

%The next statement creates the title page.
\frame{\titlepage}

%---------------------------------------------------------
%This block of code is for the table of contents after
%the title page
% \begin{frame}
% \frametitle{Table of Contents}
% \tableofcontents
% \end{frame}
%--------------------------------------------------------

{
	% \usebackgroundtemplate{\transparent{0.3}{\begin{picture}
	%     \includegraphics[height=0.7\paperheight]{cover}
	% \end{picture}    
	% }}
	\usebackgroundtemplate{
		\tikz[overlay,remember picture]
		\node[opacity=0.3, at=(current page.south east),anchor=south east, yshift=2cm,xshift=4cm] {
			\includegraphics[height=0.6\paperheight]{cover}};
	}
	\begin{frame}
		\section{\textcolor{darkmidnightblue}{Graph}}
	\end{frame}

}

\begin{frame}
	Trees are used to represent \alert{hierarchical} data. What if we want to represent non-hierarchical data?

	\begin{enumerate}
		\item A map
		\item A social network
	\end{enumerate}
\end{frame}

\end{document}
